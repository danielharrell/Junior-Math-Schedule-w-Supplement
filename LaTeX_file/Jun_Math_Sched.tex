\documentclass[10pt]{article}
\synctex=1
%\usepackage[margin=1in]{geometry}

%\usepackage{pagenote} Another package for endnotes
%\makepagenote

%\usepackage{endnotes}

\usepackage{parskip}
\usepackage{enumitem}
%\setitemize{noitemsep}
%\usepackage{pfnote}
%\usepackage{dblfnote}
%\usepackage{bigfoot}

% Comment out package below to make for endnotes
\usepackage[flushmargin,perpage]{footmisc}

\usepackage{pdfpages}
\usepackage[colorlinks=true,linkcolor={blue}]{hyperref}

\newcounter{includepdfpage}

% For endnotes
%\let\footnote=\endnote

\begin{document}
\setlist{nosep}

\begin{center}
  \textsc{Junior Mathematics Schedule}\\[5pt]
  \footnotesize{rev.~\today}\\[30pt]
  \textsc{\small{Overview}}
\end{center}

\bigskip

\begin{enumerate}[noitemsep]
  \item Selections from Galileo's \emph{Two New Sciences}, NE 68--93.%\endnote{An example of an endnote.}
  \item Essays of Leibniz, with notes:
    \begin{enumerate}
      \item ``An Approach to the Arithmetic of Infinites.''
      \item ``A New Method.''
      \item ``On Recondite Geometry.''
      \item ``True Proportion'' (optional).
      \item Two essays on the Hanging Chain (optional---\href{https://drive.google.com/file/d/1Q06ypQH26GPMVsxn8SRQVEEn79qmUxOE/view?usp=sharing}{not in manual}).
\end{enumerate}
  \item Dedekind, \emph{Continuity and Irrational Numbers}
  \item ``Cantor's Transfinite Set Theory'' (optional).
  \item Newton, \emph{Principia}:
    \begin{enumerate}
      \item Book I, Sections 1--3.
      \item Book I, Section 11: Propositions 57--69 (optional).
      \item Book III: Preface, Phenomena, Propositions 1--13.
      \item General Scholium.
    \end{enumerate}
\end{enumerate}

{\small \emph{Note}: Materials needed for Junior Mathematics 
include Galileo's \emph{Two New Sciences}, Newton's \emph{Principia}, and
\href{https://drive.google.com/file/d/13h-us1vPTZjGNSpHRpInOJR5JjXLNdWb/view?usp=sharing}%
{a first semester manual} (which includes the Leibniz essays and notes,
the Dedekind, and the account of Cantor's set theory).  
Many have also found notes on the Newton helpful, even at times necessary.  
The most commonly consulted works include Dana Densmore's \emph{Central Argument},
\href{https://drive.google.com/file/d/0BwccG5Ei3816aUtwck5GTUVRWkk/view?usp=sharing}{Robert
Bart's \emph{Notes}}, and
\href{https://drive.google.com/file/d/0BwccG5Ei3816cHlRaktYaTJRM0E/view?usp=sharing}%
{Percival Frost's \emph{Principia}}. 
The Section 11 propositions are usually done with
\href{https://drive.google.com/file/d/1c0gRZDPmndb5C2JiWYSOIiMQD-mNR2cD/view?usp=sharing}%
{``Two, Three, and Multiple Body Problems''}, written by Chester Burke.
In the more detailed schedule below, there is also occasional mention of
\href{https://drive.google.com/file/d/1PP99RsI_xEZ8rkgXSOlNBS2Wuk_CJkvN/view?usp=sharing}%
{a supplement}, which includes additional documents
that tutors and students may find useful over the course of the year.%
\footnote{A copy of the supplement is also hyperlinked to this schedule. 
If you are reading this version, then the word ``\hyperref[supple.1]{supplement}''
should appear in a different color. Pagination of the supplement appears on the upper-right
corner of each page.}
A more extensive collection of documents was put into
\href{https://drive.google.com/file/d/1iZOownvFcTZxxFA6KRutjVLpIjE74503/view?usp=sharing}%
{an online folder} by Michael Dink back in 2016--17.%
\footnote{A zip file of the folder is available at the hyperlink above,
but the folder can also be accessed at the link 
in \href{http://digitalarchives.sjc.edu/faculty}{the digital archives} copy
of Michael's \href{https://digitalarchives.sjc.edu/items/show/2677}{2016--17 archon report}.}

Some tutors have found it useful to consult the tutor-written manual we used for calculus before adopting Leibniz---the so-called \href{https://drive.google.com/file/d/15Eaxua300SdHRIPZwpTj6f9-6sOnjunB/view?usp=sharing}{Kutler manual}. 

For a list of writing assignments from recent archon reports, see \hyperref[supple.106]{page~\pageref{supple.106}} of the supplement. For a list of topics for Leibniz papers, see
\hyperref[supple.58]{page~\pageref{supple.58}} of the supplement.

}

% TODO Put links to paper topics here, along with including pages in supplement.


%\begin{center}
%	{\small {\textbf{Errata first semester manual}}}
%\end{center}
%\begin{center}
%	\small
%\begin{tabular}{r|l}
%	p28 fn5 ln2 & eliminate second ``turns'' \\
%	p31 ln6 & ``inflexion'' should be ``inflection'' \\
%	p50 ln3b & ``$VR_1$'' should be ``$V_1R$'' \\
%	p95 ln10 & delete second ``at'' \\
%	p136 \emph{Aeneid} ln3 & delete ``to'' \\
%	p143 ln2 & ``$DE_1$'' should be ``$D_1E_1$''  \\
%	p153 pr1(c) & add another close parenthesis at of expression \\
%	p158 ln4 & ``of circle'' should be ``of a circle'' \\
%	p169 ln1 & ``and chain rule'' should be ``and the chain rule'' \\
%	p169 ln13b & insert ``in'' before ``many introductory calculus books''
%\end{tabular}
%\end{center}


\newpage

\begin{center}
  \textsc{\small{First Semester}}
\end{center} 
\label{First}
{\small \emph{Note}: The following schedule 
lists thirty-six assignments for an expected forty-two classes.
Nineteen additional assignments are listed and flagged as optional.

}

\textbf{\emph{Two New Sciences}} (Galileo)
\vspace{-0.2em}
  \begin{quote}
    \small{\emph{Note}: What follows is only one approach to the Galileo.
    It does not include the so-called ``wheel paradox'' 
    or as much on the phenomenon of motion 
    as older schedules did.  
    For a two-day schedule that includes 
    the wheel paradox, along with a four-day schedule
    that includes more on motion,
    see \hyperref[Galileo]{page \pageref{Galileo}}.
    In another option, one might simply skip the Galileo,
    to allow more time for other first semester readings.
    
    } 
  \end{quote}
  \begin{enumerate}[noitemsep]
    \item Four excerpts from Aristotle's \emph{Physics}
		(in manual) and NE 77--82 (through ``than
		its division into a thousand
		parts.'').\footnote{Pagination from
			``National Edition'' (NE) found in
			the margins of the Drake-edited
			text.} \item  NE 82--89 (through
		``even after a thousand discussions?'') and
		NE 92--93 (``But it is time now'' through
		``by assuming the said composition of
		indivisibles.'').  \end{enumerate}

{\small \emph{Note}: On older schedules the Galileo was followed by Lemmas 1--11 of the \emph{Principia}, which perhaps makes a better fit with Galileo than Leibniz, especially for classes that spend more time on Galileo. In a more recent order of the past, the Galileo was followed by Leibniz, but Leibniz was followed by Newton. Either order means reading the Dedekind second semester.(See \hyperref[NewtonStart]{page~\pageref{NewtonStart}} for the start of the \emph{Principia} schedule.)

}

\textbf{An Approach to the Arithmetic of Infinites} (Leibniz)
\label{LeibnizInfinites}
\vspace{-0.2em}
\begin{quote}
	\small{\emph{Note}: In the original sequence of Leibniz readings proposed, this paper was read in tandem with a letter Leibniz wrote to Ehrenfried Walther von Tschirnhaus. See
	\hyperref[supple.86]{page~\pageref{supple.86}} of the
	supplement.}\footnote{The part of the letter relevant to the Leibniz paper begins at the paragraph ``Furthermore, as I go over the rest of your letter\ldots.'' See \hyperref[supple.92]{page~\pageref{supple.92}} of the supplement.}
\end{quote}
\begin{enumerate}[resume*]
	\item Title, from the first paragraph through the paragraph ending next
		to Note 6. Notes 2--3, 6.\footnote{Notes 1, 4--5 are optional. In the first Leibniz schedule for this reading, the instruction was to have students go to the board to work through the arguments in Notes 2, 3, and 6; and to have students prepare problem 1 in Note 3, and problems 2 and 3 in Note 6.}
	\item From the first paragraph after
		Note 6 to the paragraph ending next to
		Note 7.\footnote{This omits the
			demonstrations of the axioms following the last paragraph read.}
\end{enumerate}

\textbf{A New Method} (Leibniz)
\begin{enumerate}[resume*] \item Title and first paragraph.
		Notes 1--3.  \item Second and third
		paragraphs (``Let $a$ be a given constant
		quantity'' through ``find ways to make it
		easier.''). Notes 4--6.  \item Note 7, Part
		3. Omit proofs for negative exponents and
		root rule.\footnote{Doing Part 3 of Note 7
			before Parts 1 and 2 makes
			derivation of the rules prior to
			their application. An alternative
			approach would follow the manual in
			doing Parts 1 and 2 before Part 3.}
	\item Note 7, Part 1, Examples 1--4; Part 2,
		Problems 1--7 odd.  \item Note 7, Part 1,
		Examples 5--7; Part 2, Problems 9--23 odd.
	\item Note 7, Parts 4 and 5.\footnote{Parts 6 and 7
			of Note 7 are omitted, which involve
			the finding of tangents. An
			alternative approach would take a
			day or two to do them.} \item Fourth
		paragraph (``Once we have learned this
		\emph{Algorithm}'' through ``we could find
		it from a given property of the tangents of
		a circle'').  Notes 8--9.  \item Last
		sentence of fifth paragraph (``Let me now
		give some easier examples'') through middle
		of sixth paragraph (``things that someone
		skilled in our calculus will henceforth be
		able to produce in three lines''). Notes
		12--19.\footnote{The first example of the
			calculus that Leibniz gives (fifth
			paragraph) is
			skipped.} \item Later in sixth
		paragraph (``As an appendix'') to the end of
		the paper.  Note 22, Parts 1--5.\footnote{An
			example in paragraph six (``I will
			show this with yet another
			example'') is skipped. Also, Parts
			1--5 may be too much for a single
			day. An alternative approach might
			divide this assignment over two
			days: Parts 1--2; 3--5. One might
			also replace Note 22 with a shorter
		tutor-written note, ``De Beaune’s problem to
		Descartes, solved by Leibniz''---see
		\hyperref[supple.14]{page~\pageref{supple.14}}
		of the supplement. On
	\hyperref[supple.20]{page~\pageref{supple.20}} of
	the supplement can also be found ``How to approach an
	exact value for $e$,'' based on Theorem 3 in Part
	3.} \item Note
		22, Part 6.  \end{enumerate}
	 
\textbf{On Recondite Geometry} (Leibniz)
\begin{enumerate}[resume*] \item Title, first three
		sentences of first paragraph (through
		``those who try to prove the impossibility
		of quadrature neglect this distinction'').
		Notes 1--2.  
	\item Fourth and fifth sentences of first paragraph
		(``He recognizes with me that the figures''
		to ``since it supplies the best remedy
		against irrationalities''). Third, fourth,
	       and fifth paragraphs (``Further, to say
	       something more useful here'' to ``thus are
	       not comprehended by means of algebraic
	       equations''). Notes 3--10.\footnote{The rest
		       of the first paragraph and the second
		       paragraph are skipped. A brief
		       elaboration of Note 8 can be found on
		       \hyperref[supple.21]{page~\pageref{supple.21}}
		       of the supplement; and 
	       on
	       \hyperref[supple.22]{page~\pageref{supple.22}},
a fuller treatment of steps in Notes 8 and 9,
	       from an older version of the notes.}
       \item Sixth paragraph (``Furthermore, because hardly
	       anything can be imagined that is more
	       useful'' through ``innumerable other things
	       are also derived from this''). Notes 11--17.	       
       \item Seventh and eighth paragraphs (``Finally, so
	       that I may not seem to ascribe too much to
	       myself'' to ``innumerable transfigurations
	       and equipotencies of figures may arise from
	       this very same thing''). Note
	       18.\footnote{The seventh paragraph might be
		       skipped. And the final paragraph of
		       the paper goes
		       unread.}
       \item Note 19, Parts 1 and 2.\footnote{These parts
		       of Note 19 are sometimes found
		       wanting, especially the
		       formulation and proof of the first
		       fundamental theorem. For a different
		       approach to both theorems, see 
		       \hyperref[supple.28]{page~\pageref{supple.28}}
		       of the supplement.} 
       \item Note 19, Part 3.\footnote{See
		       \hyperref[supple.30]{page~\pageref{supple.30}}
		       of the supplement for an alternative
		       account of algebraic quadratricies,
		       which existed in the manual for two
		       years, is 
		       closer to what is found in calculus
		       textbooks, and is preferred by some
		       tutors.}  
       \item Note 19, Part 4 through Example 4.
       \item Note 19, rest of Part 4.
       \item Note 19, Part 5 through ``Transcendence of the
	       logarithmic line.''
       \item Note 19, rest of Part 5.
\end{enumerate}

{\small \emph{Note}: Given
       time and motivated students, at this point one might
       study the hanging
	chain problem in two additional papers of Leibniz
	(\href{https://drive.google.com/file/d/1Q06ypQH26GPMVsxn8SRQVEEn79qmUxOE/view?usp=sharing}{not in manual}).
	See \hyperref[LeibnizHang]{page
		\pageref{LeibnizHang}} for a four-day
	schedule of this. A shorter option: Leibniz papers that used to be read on the isochronic line (not in manual; consists of separate \href{https://drive.google.com/file/d/13w3nTIpDN1F0sgImLWWRaLF5vg2cF6BL/view?usp=sharing}{readings} and \href{https://drive.google.com/file/d/140fF6SF7omRfQ-sFkhHiBlaNOiBiNvUq/view?usp=sharing}{comments}). See \hyperref[LeibnizIso]{page \pageref{LeibnizIso}} for a two-day schedule.
	
	}
	
	\textbf{Mathematics for Newton's physics}
	(tutor-written)
	\vspace{-0.3em}

\ ``Functional Notation and
	Calculus''
	\vspace{-0.3em}

\begin{quote}
{\small \emph{Note}: A related text on functions, sometimes
read in the past, is Frege's \href{https://drive.google.com/file/d/14ZN_SpyI3nYJz9Ru-C7RXrOKIAnukjTM/view?usp=sharing}{``What is a Function?''}

}
\end{quote}
	
	

\begin{enumerate}[resume*]
	\item First two sections: ``Functions and
		derivatives''; ``Finding derivations in functional
		notation.'' Problem 1 optional. 
	\item Last two sections: ``The method of
		substitution and the chain rule'';
		``Functions of more than one variable and
		partial derivatives.'' Problems 2--10 optional. 
\end{enumerate}

\vspace{-0.3em}
\ ``Calculus and Newtonian Physics''
\begin{enumerate}[resume*]
	\item First paragraph and first two sections:
		``Velocity and force''; ``The phenomenon and
		force of one falling body.'' Problems 1--5
		optional.
	\item Last section: ``Projectile motion''. Problems
		1--4 optional.
\end{enumerate}
{\small \emph{Note}: Given time and interest, at this point one
might read another Leibniz paper (in manual): 
	``On the True Proportion, Expressed
	in Rational Numbers, of a Circle to a Circumscribed
	Square.'' See \hyperref[LeibnizProp]{page
		\pageref{LeibnizProp}} for a
	two-day schedule for this. A one-day alternative is
	to read the defense of his calculus given by Leibniz
	in two letters, or in an early manuscript; see
	\hyperref[supple.46]{page~\pageref{supple.46}} of
	the
supplement for the first and
\hyperref[supple.51]{page~\pageref{supple.51}} for the
second.

}

\textbf{\emph{Continuity and Irrational Numbers}} (Dedekind)\footnote{Thanks to Andr\'e Barbera, the supplement contains two articles that appeared in \emph{Energeia} in 1965 about Dedekind and Euclid. The first, written by Sam Kutler, can be found on \hyperref[supple.100]{page~\pageref{supple.100}}; the second, by Eva Brann, can be found on \hyperref[supple.102]{page~\pageref{supple.102}}. There is also a longer \href{https://drive.google.com/file/d/14ZP9-NtLSKE-jeJbgjqxgkxo7QWGJVR1/view?usp=sharing}{Kutler essay
on Dedekind}.}
% TODO Mention Kutler article here.
\vspace{-0.2em}
\begin{quote}
\small{\emph{Note}: Another option, found on older schedules, is to read Dedekind second semester, after Newton. This means starting \emph{Principia} first semester, either before or after Leibniz. (The \emph{Principia} schedule begins on \hyperref[NewtonStart]{page~\pageref{NewtonStart}}.)

} 
\end{quote}
\begin{enumerate}[resume*]
	\item Untitled Preface. Note 1.\footnote{Note 1 is
			long and not directly connected to
		       Dedekind. An alternative approach
	       would delay or even skip this note, and
	       combine this day with the following one. If
       delayed one might read it second semester, after the
first two lemmas of \emph{Principia}. (See
\hyperref[Dedekind]{page~\pageref{Dedekind}} for more on
	this.)}	       
	\item Chapters I--II.
	\item Chapter III. 
	\item Chapter IV: first through beginning of sixth
		paragraph (``Hence the square of every
		rational number $x$ is either $<D$ or
		$>D$''). Note 2 through step (16).
	\item Chapter IV: sixth through ninth paragraph
		(stopping just before paragraph that begins
		``In order to obtain a basis for the orderly
		arrangement of all \emph{real}, i.e., of all
		rational and irrational numbers''). Rest of
	       Note 2.	
       \item Rest of Chapter IV.
       \item Chapter V.
       \item Chapter VI.\footnote{Chapter VII is omitted but
		       could be done with an additional
		       class.}
\end{enumerate}
{\small\emph{Note}: Given time and interest, at this point one
might study Cantor's set theory. Alternatively, this study
might be done second semester, before or after the Newton.
See \hyperref[Cantor]{page~\pageref{Cantor}} for a five-day 
schedule. One way to end the Leibniz-Dedekind-Cantor sequence
is through a brilliant essay by Jos\'e Benardete, \href{https://drive.google.com/file/d/14YTO3e3k-1osmcZWnsmN2aEGdqNfefOc/view?usp=sharing}{``Continuity and Theory of Measurement.''}

}
\bigskip
\begin{center}
	\textsc{\small{First semester: optional readings}}
\end{center}
\textbf{\emph{Two New Sciences}} \label{Galileo}
\begin{enumerate}[noitemsep]
	\item NE 68 to
	middle of 72 (through ``inappropriately
	did add'') and from just before 92 to just before 97
	(``But it is now time'' to ``and
	natural materials.''). 
\item Four excerpts from
	Aristotle’s \emph{Physics} (in manual). NE 77--82 (through
	``than its division into a thousand parts.'')
	and NE 92--93 (``But it is time now'' through
	``by assuming the said composition of
	indivisibles.''). 
\end{enumerate}
\textbf{\emph{Two New Sciences} and \emph{Physics}}
\begin{enumerate}[noitemsep]
	\item Three excerpts from the \emph{Physics} (not
		in manual---see 
		\hyperref[supple.1]{page~\pageref{supple.1}}
		of supplement):
		232b20--233a32; 239b5--240a18;
		263a11--263b9. Definition of motion from
		\emph{Physics} III:1. 
	\item Optional: Excerpts from Bergson's \emph{Creative Evolution} and
		\emph{Matter and Memory} (not in
		manual---see
		\hyperref[supple.7]{page~\pageref{supple.7}}
		of supplement).
	\item Optional: Excerpt from Aristotle's \emph{Mechanical
			Problems} (not in manual---see
		\hyperref[supple.4]{page~\pageref{supple.4}}
		of supplement).
	\item Four (or more) days on NE 68--96, along with
		excerpts from Aristotle's \emph{Physics} in
		manual.
\end{enumerate}
\textbf{On the True Proportion}
	\label{LeibnizProp}
	\begin{enumerate}[noitemsep]
		\item First five paragraphs, ending with ``This is
	the kind of quadrature I am presenting here.'' Notes
	2--9. \item Continuing from sixth paragraph
	through middle of ninth paragraph (from
	``Accordingly I found'' through the indented and
	italicized ``If the area of the
	inscribed square is $2/4$, etc.''). Notes
	11--12.\footnote{The rest of the paper is omitted.}
\end{enumerate}
\textbf{The Hanging Chain problem} \href{https://drive.google.com/file/d/1Q06ypQH26GPMVsxn8SRQVEEn79qmUxOE/view?usp=sharing}{(not in manual)}
\label{LeibnizHang}

\ ``On the Line in which a Heavy Body Bends
by its own Weight''
\begin{enumerate}[noitemsep]  \item Title.
		First through third paragraphs. The
		italicized ``primary Problems'' 1--6 but not
		their solutions. Notes.  \end{enumerate} 

\ ``A Solution of the Problem First Proposed by
Galileo'' 
\begin{enumerate}[resume*]
	\item Skip first two paragraphs; begin with ``An analysis of the
	chain problem.'' Read through equation~11. Notes
	1--11. 
\item Read through equation 20. Notes
	12--18. \item Read to the end. Remaining
	Notes.
	\end{enumerate} 
\textbf{The Isochronic Line Problem}
\label{LeibnizIso}
\vspace{-0.2em} 
\begin{quote}
{\small Not in manual---separate \href{https://drive.google.com/file/d/13w3nTIpDN1F0sgImLWWRaLF5vg2cF6BL/view?usp=sharing}{readings} and \href{https://drive.google.com/file/d/140fF6SF7omRfQ-sFkhHiBlaNOiBiNvUq/view?usp=sharing}{comments}. Schedule follows pagination in text.

}
\end{quote}

\vspace{-0.2em}

\begin{enumerate}
\item ``A Brief Demonstration" pages 41--43. ``On the Isochronic Line," pages 44--47.
Comment 1, pages~145--146.  The other comments are optional.

\item Leibniz, ``An undated manuscript on the isochronic line," pages 48-51.
Omit the ``Problem" and its demonstration.
\end{enumerate}


\textbf{Cantor's Transfinite Set Theory}
\label{Cantor}
\vspace{-0.2em}
\begin{quote}
	{\small\emph{Note:} What follows is a five-day schedule 
using the tutor-written account 
in the manual. One could reduce this to one or two days,
devoted only to the proofs for the denumerability of
the rationals and non-denumerability of the reals,
using either 
Sections 9 and 12 in this treatment or
a different text for the purpose  (such as
the sections from Courant and Robbins, \emph{What
	is Mathematics?} found on
\hyperref[supple.59]{page~\pageref{supple.59}} of the
supplement).
There is also a separate copy of the text with \href{https://drive.google.com/file/d/14VyXI_JIWel6-rEA67aYdqD_CrFCRLXo/view?usp=sharing}{a supplement of notes}.

}
\end{quote}
\begin{enumerate}[noitemsep]
	\item Sections 1--7.
	\item Sections 8--11.
	\item Sections 12--13.
	\item Sections 14--15.
	\item Sections 16--17.
\end{enumerate}

\bigskip

\begin{center}
\textsc{\small{End of First Semester}}
\end{center}

%\theendnotes

\newpage
\begin{center}
	\textsc{\small{Second Semester}}
\end{center}
{\small \emph{Note}: The following schedule lists 
	thirty-six assignments for an expected forty-two
	classes. Five additional
	assignments are listed and flagged as optional.
	
	}

\textbf{\emph{Principia}} (Newton)
\phantomsection\label{NewtonStart}
\vspace{-0.5em}
\begin{quote}
{\small\emph{Note}: There is no assigned place on this
	schedule for a helpful, even necessary, review of
	the definitions and laws that precede the lemmas and
	read in Junior Laboratory first semester. This might
	be assigned over winter break along with Lemmas 1
	and 2. For notes on this material, from the Junior
	Laboratory manual, see
	\hyperref[supple.64]{page~\pageref{supple.64}} of
	the supplement. There is also a website associated
	with the Integral Program of St.\ Mary's College of
	California that permits viewers to interact with
	various lemmas and propositions of the \emph{Principia}.
	The current address of the website is: \href{https://www.geogebra.org/m/YlONul2N}{https://www.geogebra.org/m/YlONul2N}.
	
	}
\end{quote}
\begin{enumerate}[noitemsep] \item
		Lemmas 1 and 2.  \end{enumerate}
		\vspace{-0.5em}
\begin{quote} {\small \emph{Note}: At this point, in a
		practice of long-standing, one might devote
		the next class (or even two classes) to
		Archimedes' \emph{On the Measurement of the
			Circle} (in manual) and Euclid's
		\emph{Elements} X.1 and XII.2, along with
		the second paragraph of the scholium after
		Lemma 11. One might also read or review Note
		1 to the Dedekind.  \label{Dedekind}
		
		}
\end{quote}
\begin{enumerate}[resume*] \item Lemma 3 with
		corollaries.  \item Lemma 4 with corollary.
		Lemma 5.  \item Lemmas 6 and 7.\footnote{One
			might assign a review of
			\emph{Elements} III.16 with Lemma 6
			and allow two days here.} \item
		Corollaries to Lemma 7. Lemma 8.  \item
		Lemma 9.  \item Lemma 10 with Corollaries 3
		and 4.\footnote{Corollary 4 is invoked in
			I.6 and relevant to III.4. This is
			also sometimes assigned with a
			review of Prop.~II of ``On naturally
			accelerated motion'' from \emph{Two
				New Sciences} (NE 208 ff).}
	\item Lemma 11 through Case 1.\footnote{A supplement
			here is useful that discusses finite
			curvature. Possible texts: ``Note on
			Curvature'' in Junior Lab manual
			(see
			\hyperref[supple.74]{page~\pageref{supple.74}}
			of supplement); Bart's \emph{Notes}
			(36 ff); Densmore's \emph{Central
				Argument} (74; 109);
			Comenetz's \emph{Calculus} (332 ff).
			Henry Higuera has also written
			\href{https://drive.google.com/file/d/1YZfQPV6poJMzZvZ3DyGfJopGkzy7oe7K/view?usp=sharing}{notes
			that start with Lemma 11 and
			end with Prop.~17}.}
	\item Lemma 11,
		Cases 2 and 3.  Begin corollaries.  \item
		Finish corollaries and read entire scholium
		to Lemma 11.  \end{enumerate}
		\begin{quote}
		\vspace{-0.5em}
	{\small \emph{Note}: More than one archon report
		suggests spending more time on the Book I
		propositions, especially the earliest ones,
		than indicated below (although in earlier
		times 7--9 were skipped).  There is also
		disagreement about doing or omitting ``same
		otherwise'' proofs in general; this schedule
		draws no attention to them. Unlisted
		corollaries and scholia are also assumed
		optional.
		
		} \end{quote}
\begin{enumerate}[resume*] \item Proposition 1.  \item
		Corollaries 1, 2, and 4 of
		Prop.~1.\footnote{Corollary 1 of the Laws
			might be reviewed here.} \item
		Proposition 2.  \item Proposition 3.
		Scholium after 3.  \item Proposition 4.
		Corollaries 1--7, 9.\footnote{Corollary 9 is used in III.4, an important proposition.}  \item Proposition
		5.\footnote{This proposition is regarded as
			optional on some older schedules.}
	\item Proposition 6 and Corollary 1.\footnote{A
			shorter alternative would be
			a class on Proposition 6 with Corollaries 1 and
			5, without a second class.}  \item
		Remaining corollaries to Prop.~6.\footnote{In his 2018--19 archon report, Andr\'e Barbera notes that in Prop.\,6, Cor.\,3, ``Newton seems to have Euclid III.32 in mind. I do not see
		this indicated in either Bart or Densmore.''}
		  \item
		Proposition 7.\footnote{\emph{Elements}
			III.32 and III.36 might be reviewed
			here.} \item Propositions 8.
		Scholium after 8. Proposition
		9.\footnote{Proposition 8 is regarded as
			optional on some older schedules.}
	\item Lemma 12. Proposition 10 with corollaries.
		Scholium after 10.\footnote{\emph{Conics}
			I.15, I.21, and I.37 might be
			reviewed here. For an interesting
			tutor note (Joe Sachs) on the
			sequence of propositions from 4 to
			10, see
			\hyperref[supple.77]{page~\pageref{supple.77}}
			of the supplement.} \end{enumerate}
			\vspace{-0.5em}
\begin{quote} \small{\emph{Note}: More than one archon
		suggests spending less time on the remaining
		propositions of Bk I than indicated here (11
		and 12 are sometimes combined into a single
		day; while 14–16 are sometimes only
		enunciated). A usual practice in earlier
		years omitted 16 and 17, and some tutors
		have doubted the necessity even of 14 and
		15, which go unused in (what we read of) Bk
		III. Another time-saving suggestion of the
		past: treat 12 and 13 lightly, or even
		omit.}  \end{quote}
\begin{enumerate}[resume*] \item Proposition
		11.\footnote{\emph{Conics} III.48 and III.52
			might be reviewed here.} \item
		Proposition 12.\footnote{\emph{Conics}
			III.51 might be reviewed here. Also,
			to save time, one could simply draw
			the diagram to show the analogy with
			the proof in 11 for the ellipse.}
	\item Lemmas 13 and 14. Proposition 13 with
		corollaries.\footnote{\emph{Conics} III.45,
			I.46, and I.49 (with the
			translator's note for 49) might be
			reviewed here. Two notes on Corollary 1 can be
			found on \hyperref[supple.96]{page~\pageref{supple.96}} and \hyperref[supple.98]{\pageref{supple.98}} of the supplement.} \item Propositions
		14 and 15, with their corollaries.  \item
		Proposition 16.  \item Proposition 17.\footnote{A
		note on Prop.\,17 can be found on \hyperref[supple.99]{page~\pageref{supple.99}} of the supplement.}
	\item Review and discussion of Propositions 1--17.
\end{enumerate}
\vspace{-0.5em} 
\begin{quote} \small{\emph{Note}: This
		schedule follows Newton's advice at the
		beginning of Book III, moving straight to
		Book III after Proposition 17 of Book I. But
		a common alternative---and some tutors would
		argue for its importance---is to precede
		Book III with a study of Propositions
		57--69, since 69 is needed for III.7 and
		57--69 form a sequence in itself (section 11
		of Book I). A five-day schedule for this
		sequence can be found on page
		\pageref{Newton}.} \end{quote}
\begin{enumerate}[resume*] \item Preface. Rules.  \item
		Phenomena. Propositions 1 and
		2.\footnote{This schedule gives perhaps the
			most compressed assignment possible
			for the six phenomena that precede
			the propositions in Book III. One
			strategy is refer back to the
			phenomena as they are invoked in the
			proofs for the propositions. But
			older schedules devoted three days
			to the phenomena alone (grouping the
			six in pairs). And a look at
			Densmore's treatment in her
			\emph{Central Argument} (devoting
			nearly fifty pages to them) gives a
			sense of how much one can make of
			the phenomena on their own.}
	\item Propositions 3 and 4. Scholium.\footnote{Proposition 4 would be the focus this day, and arguably deserves a day of its own.}
	\item Proposition 5. Corollaries. Scholium.
	\item Proposition 6. Corollaries.\footnote{On older
			schedules, Propositions 5 and 6 were done
			together, with all the corollaries
			of 5 and Corollaries 1, 2, and 5 of
			6.} 
	\item Proposition 7. 
	\item Propositions 8--13. Hypothesis after
		10.\footnote{A strategy for this assignment
			is to restrict the reading and
			discussion to the enunciations of the
			propositions. Another approach would
		focus on Propositions 8 and 13. Still
		another approach might spend a few days on
		the sequence of propositions in Book I
		(71--76) that lead to III.8, and as if culminating in 8.}
	\item General Scholium after Book III (at end of
		\emph{Principia}).
	\end{enumerate}
	\begin{quote}
		{\small \emph{Note}: At this point, given
			time (or even with the
			General Scholium in a single 
			asssignment), one might read an excerpt from
		Newton's \emph{Optics} (Query 31) that bears
	on what he writes in the General Scholium. See
	\hyperref[supple.79]{page~\pageref{supple.79}} of the
		supplement.
		
		}
	\end{quote}	
	\begin{center} \textsc{\small{Second
				semester: optional
				readings}} \end{center}
	\textbf{\emph{Principia}, Book I, Section 11}
	\label{Newton} 
	\begin{quote} \small{\emph{Note}: In
			addition to commentary on certain of
			these propositions in both the
			Densmore guidebook and the Bart
			manual, there is \href{https://drive.google.com/file/d/1c0gRZDPmndb5C2JiWYSOIiMQD-mNR2cD/view?usp=sharing}{a separate set of
			notes} for this sequence written by
			Chester Burke.} \end{quote}
	\begin{enumerate}[noitemsep] \item Introduction.
			Proposition 57.  \item Proposition
			58.  \item Proposition 64. Read
			through Proposition 65.  \item
			Proposition 66. Corollaries 1--6.
			Browse through all the corollaries.
		\item Proposition 69 with its corollaries.
			Scholium after 69.  \end{enumerate}
			\bigskip
			\begin{center} \textsc{\small{End of Second Semester}} \end{center}
% For endnotes page
%\newpage
%\begingroup
%\parindent 0pt
%\parskip 2ex
%\def\enotesize{\normalsize}
%% \pdfbookmark{Endnotes}{Endnotes}
%\theendnotes
%\endgroup

	\newpage \setcounter{page}{1}
	\includepdf[pages=-,link,linkname=supple,pagecommand={\thispagestyle{headings}\refstepcounter{includepdfpage}\label{supple.\theincludepdfpage}}]{supplement.pdf}
\end{document}
