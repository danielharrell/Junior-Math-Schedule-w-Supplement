\documentclass{article}
\title{Writing assignments indicated in recent archon reports}
\date{}
\begin{document}
\maketitle
\section{From the 2014 report}
We all assigned calculus problems from among those in the manual, usually just the odd numbers, to be turned in by students. We would then go over in class any that presented difficulties.

Most of us assigned one paper (4--7 pages) each semester, for the most part allowing students to choose their topics. Jeff Smith assigned two ``short-answer'' papers for semester, for a total of 8 pages per semester. Louis Petrich gave students alternative due dates to better accommodate their other assignments. In the second semester, Steve Crockett encouraged them to write about a difficult proposition, a proposition the class hadn't done, or Query 31. In the first semester, I asked students to trace one question from Euclid through Descartes to Leibniz.

Some of us felt in retrospect that more focused paper topics and/or additional shorter assignments might have been valuable. Given the importance of Prop.~VI, Cor.~I and the confusions about it which emerged in my students' Newton papers, I might in future ask for a straightforward exposition of its proof.

\section{From the 2015 report}
Most of us assigned at least two papers in each semester. In the first semester I assigned one paper on Leibniz's calculus and another on Dedekind. In the second I assigned a paper on the difference between Leibniz's and Newton's approaches to the calculus and a paper on a topic of choice in the Principia. Some tutors used suggested questions written by me or by Louis Petrich. One of the most fruitful Newton topics several students took up in my class was the curious ``solid'' that Newton introduces in the first corollary of proposition 6 and that appears throughout the search for force laws.
\section{From the 2016 report}
	\subsection{First Semester}
	All tutors so far as I know assigned all the problem
	sets in the manual as written assignments. One tutor
	also had an oral for each student, on a topic of the
	student's choice, and otherwise required daily
	writing of a paragraph either explicating or
	questioning each reading. One tutor gave students four
	opportunities (one with Galileo, two with Leibniz,
	one with Dedekind; all with suggested questions) to
	write two papers, 4--6 pages in length. The
	remaining tutors assigned either one paper or two,
	of similar length. One tutor assigned a Dedekind
	paper on the question ``Why does mathematics, a
	human invention, at least in Dedekind's conception
	of it, have properties?'' Another tutor assigned a
	Leibniz paper on Snell's Law in ``New Method''
	requiring an account of the mathematics and then a
	reflection on Leibniz's claims about final
	causality. Other tutors assigned their papers on
	Leibniz, giving students their choice of topic,
	while one tutor suggested questions such as:  
	``What is $dx$?''; ``What does
	Leibniz understand by a sum ($\int$)?''; ``What is
	`recondite geometry'?''; ``How is Leibniz able to claim
	that to find a tangent `is to draw a straight line
	joining two points on a curve'?'' 

	\subsection{Second Semester}
	All tutors so far as I know required demonstrations
	at the board from volunteers for at least some
	Newton propositions. (It is doubtful that one could
	make it through the schedule by requiring such
	demonstrations for all propositions.) And all
	the other writing assigned was presumably on
	\emph{Principia}, the second semester text.
	The tutor who assigned an oral and daily writing
	first semester required a 10 page paper near the end
	of second semester, preceded by several written
	updates on progress. Another tutor continued his
	practice of giving students four opportunities to
	write two essays, each 5 pages in length. Other tutors
	also assigned two papers of similar length, with one tutor
	dividing the papers between the initial lemmas and
	propositions 1--17 in Book I. So far as I know, all
	of these papers were on topics of the student's
	choice. 
	
	
\section{From the 2017 report}
Most tutors did some mix of shorter papers focused on the details of an individual proposition or other technical issue and longer papers reflecting on the arc of some sequence or the larger issues at stake in an author’s project, generally amounting to no less than the equivalent of two 4-5 page papers per semester.


\section{From the 2021 report}

First, what I assigned each semester: Fall---an initial short paper on what Leibniz means by the letter ``d,'' followed by a longer paper, due end of term, on a topic and reading of the student's choice. (Most picked Leibniz; a few wrote on Dedekind; none  on Galileo.) Spring---a longer paper on Newton, due end of term. (Most wrote on Book III.) 

Next, emailed reports from three tutors about their assignments:
\begin{quotation}{\small In my tutorial, the written work for the first semester consisted of homework assignments (the problems in the manual) and one major paper on Leibniz. The written work for the second semester consisted of a very short paper on Newton's lemmas and a major paper on some one or more of Propositions 1--17 of Book I.}  
\end{quotation}

	\begin{quotation}
		{\small On writing assignments: I assigned a paper each semester. 

First semester I assigned a paper on ``New Method''. Students were to come up with their own question, though I supplied some prompts. I also encouraged them to see whether they could draw any connections between their Leibniz seminar readings and the ``technical'' approaches of ``New Method''. I think this worked pretty well; I really did stress to them that I didn't just want ``philosophy'' papers, but they should dive into the details of the math. First semester I also had students regularly hand in homework---some of the problems in the manual. 

Second semester I assigned a paper on the Lemmas. I encouraged students to craft a question based on their reading of the Scholium to the Lemmas, and then explore it by discussing one or two Lemmas in detail. One advantage this had is it meant specific students became ``experts'' on certain Lemmas, and they were helpful once we got into the propositions.}
\end{quotation}
%\vspace{-10pt}
%{\centering{\rule{50pt}{0.5pt}}\par}
%\vspace{-6pt}	
	\begin{quotation} {\small As for writing assignments, I'll just report that I assigned one 5 page paper per semester: a Leibniz paper towards the end of our time with him (the timing of which worked well, I thought), and a Newton paper after we finished the Lemmas (this worked well in itself, but I felt something of a lack for not having a paper on the remainder of the \emph{Principia}, although to be honest that was strategically intended).} \end{quotation}
\end{document}